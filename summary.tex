
\begin{summary}

	This report presents a study on autonomous exploration on metal structures, focusing on the evaluation of three exploration strategies.
	The objective of this study was to develop effective methods allowing autonomous robots to explore a metal structure, in a complete and efficient way, in search of corrosion points.
	The three strategies evaluated include \textit{Roll Painting}, \textit{Nordic Skiing} and \textit{Polygonal Investigation}, all three based on occupancy grids.
	% The \textit{Roll Paint} strategy is a simple but robust approach, which exhaustively covers the search space with rectilinear trajectories and simultaneous movements of the robots.
	% The \textit{Nordic Skiing} strategy is a more complex approach, which introduces a phase shift in the movement of the different robots.
	% The \textit{Polygonal Investigation} strategy tries to improve the result of the previous strategies by investigating around the detected corrosion points.
	The experiments were carried out in simulation using Gazebo and a crawler model developed for the European project BugWright2.
	These robots are notably equipped with UGW sensors, specific to our problem.
	The performances of the different strategies were evaluated in terms of investigation time and accuracy of the mapping obtained.
	The results obtained demonstrated the effectiveness of each strategy.
	The \textit{paint roller} strategy allowed for a quick but imprecise investigation.
	The \textit{Nordic skiing} strategy allowed a slow but rather precise investigation.
	Finally, the \textit{polygonal investigation} strategy made it possible to combine the advantages of the other two strategies by allowing a less slow and more precise investigation than the previous one.
	Future perspectives include improving the polygonal exploration strategy by developing more robust methods for collision management.
	In addition, the extension of this study to experiments with several teams of robots constitutes an interesting avenue for further accelerating the investigation time.
	This study contributes to research in autonomous investigation and provides indications for the development of effective investigation systems in corroded metallic environments.
	The results obtained have important implications in various fields, such as service robotics, space exploration and environmental monitoring.

\end{summary}