
\chapter{Conclusion}

In conclusion, this study made it possible to implement and evaluate three strategies for performing autonomous muli-robot exploration in complex environments.
The results obtained demonstrate the effectiveness of these approaches in solving the problem of inspecting metal surfaces and highlight their respective characteristics.

The first strategy, the \textit{Roller Painting} strategy, made it possible to obtain satisfactory results in very short exploration times.
The distance between the robots can be adjusted to optimize the results depending on the density of the inferred corrosion zones.
This strategy is particularly suitable for use prior to the \textit{Polygonal Investigation} strategy, because it allows you to quickly get a rough view of potential corrosion areas.

The second strategy, the \textit{Nordic Skiing} strategy, provided better results than the \textit{Roller Painting} strategy, but with longer exploration times.
This approach is more robust than the roller painting strategy, because it varies the orientation of the emitted and received \gls{ugw} rays, thus making it possible to approach the corrosion zones more finely.

Finally, the third strategy, the \textit{Polygonal Investigation} strategy, yielded the best results, in terms of accuracy.
This reactive strategy makes it possible to refine the location of the corrosion zones based on the results of one of the two previous strategies.
However, this strategy is more sensitive to collisions between robots, which can lead to less satisfactory results in some cases.

In terms of prospects, several areas of development can be envisaged.
First of all, it would be interesting to deepen the polygonal exploration by looking for more robust methods for the management of collisions with the different robots.
This would make this strategy more reliable and usable in a wide range of complex environments.
The extension of this study to experiments with several teams of robots is also a promising avenue for further accelerating the process of exploration and investigation.
Finally, the deployment of this approach on a real system would be an important step to validate the results obtained in simulation and demonstrate the effectiveness of this approach in an industrial context.

In conclusion, this study has highlighted the advantages and limitations of three autonomous exploration strategies. The results obtained open the way to many development prospects, particularly with regard to improving the efficiency and robustness of existing approaches. These advances could have a significant impact in various fields, such as service robotics, space exploration or environmental monitoring.

